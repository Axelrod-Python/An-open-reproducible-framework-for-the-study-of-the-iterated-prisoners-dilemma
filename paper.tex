\documentclass{article}

% Nicer bibliography management
\usepackage[backend=biber, backref=true, firstinits=true, url=true, isbn=true]{biblatex}
\addbibresource{references.bib}
%-----------------------------

% Lazy page formating (we can adjust when submitting)
\usepackage{fullpage}
\usepackage[parfill]{parskip}
%-----------------------------

\usepackage{amsmath}
\usepackage{hyperref}
\usepackage{booktabs}

\title{An open reproducible framework for the study of the iterated prisoner's
dilemma}

\author{Owen Campbell\\
        \and
        Marc Harper\\
        \and
        Vincent Knight\\
        \and
        Karol Langner\\
}

\date{\today}

\begin{document}

\maketitle

\section{Introduction}\label{sec:introduction}

This paper introduces a software package: the Axelrod python library
\cite{Axelrod-Pythonprojectteam2015}. The Axelrod-Python project has the
following stated goals:

\begin{itemize}
    \item To enable the reproduction of previous Iterated Prisoner's Dilemma
    research as easily as possible.
    \item To produce the de-facto tool for any future Iterated Prisoner's
    Dilemma research.
    \item To provide as simple a means as possible for anyone to define and
    contribute new and original Iterated Prisoner's Dilemma strategies.
\end{itemize}

This library is motivated by an ongoing discussion in the academic community
about reproducible research \cite{Crick2014a, Prlic2012, Sandve2013,
Hong2015a}. The library is:

\begin{itemize}
    \item Open: all code is released under an MIT license;
    \item Reproducible: at the time of writing there is an excellent level of
        integrated tests with 99.59\% coverage;
    \item Documented: all features of the library are documented for ease of
        use.
    \item Extensive: over 100 strategies are included.
\end{itemize}

Before describing the package in more detail in
Section~\ref{sec:description-of-axelrod-python}, an overview of some previous
Iterated Prisoner's Dilemma research will be given.

\subsection{Review of the literature}\label{sec:review}

As stated in~\cite{Bendor1991}: ``\textit{few works in social science have had
the general impact of [Axelrod's study of the evolution of cooperation]}''.  In
1980, Axelrod wrote two papers:~\cite{Axelrod1980a,Axelrod1980b} which
described a computer tournament that has been at the origin of a majority of
game theoretic work~\cite{Banks1990, Bendor1991, Boyd1987, Chellapilla1999,
DavidB1993, Doebeli2005, Ellison1994, Gotts2003, Hilbe2013, Isaac2008,
Kraines1989, Lee2015, Lorberbaum1994, Milgrom1982, Molander1985, Murnighan2015,
Press2012, Stephens2002, Stewart2012}. As described in~\cite{Bendor1991} this
work has not only had mathematical impact but has also led to insights in
biology (for example in~\cite{Stephens2002}, a real tournament where Blu Jays
are the participants is described) and in particular to the study of evolution.

The tournament is based on an iterated game (see~\cite{Maschler2013} or similar
for details) where two players repeatedly play the normal form game of
(\ref{equ:one-shot}) in full knowledge of each others playing history to date.
An excellent description of the \textit{one shot} game is given
in~\cite{Gotts2003} which is paraphrased below:

Two players must choose between \textit{Cooperate} (\(C\)) and \textit{Defect}
(\(D\)):

\begin{itemize}
    \item If both choose \(C\), they receive a payoff of \(R\)
        (\textbf{R}eward);
    \item If both choose \(D\), they receive a payoff of \(P\)
        (\textbf{P}punishment);
    \item If one chooses \(C\) and the other \(D\), the defector receives a
        payoff of \(T\) (\textbf{T}emptation) and the cooperator a payoff of
        \(S\) (\textbf{S}ucker).
\end{itemize}

\begin{equation}
    \begin{pmatrix}
        R,R & S,T\\
        S,S & P,P
    \end{pmatrix}\quad\text{such that } T>R>P>S \text{ and } 2R > T + S
    \label{equ:one-shot}
\end{equation}

The game of (\ref{equ:one-shot}) is called the Prisoner's Dilemma. Numerical
values of \((R,S,T,P)=(3,0,5,1)\) are often used in the literature. Axelrod's
tournaments (and further implementations of these) are sometimes referred to as
Iterated Prisoner's Dilemma (IPD) tournaments, an overview of published tournaments is
given in Table~\ref{tab:tournaments}.

\begin{table}[!hbtp]
    \begin{center}
        \begin{tabular}{ccccc}
            \toprule
            Year     & Reference           & Number of Strategies & Type     & Source Code\\
            \midrule
            1979     & \cite{Axelrod1980a} & 13                   & Standard & Not immediately available\\
            1979     & \cite{Axelrod1980b} & 64                   & Standard & Not immediately available\\
            1991     & \cite{Bendor1991}   & 13                   & Noisy    & Not immediately available\\
            2002     & \cite{Stephens2002} & 16                   & Wildlife & Not applicable\\
            % Is this worth including? If so I think we need more.
            \bottomrule
        \end{tabular}
    \end{center}
    \caption{An overview of published tournaments}\label{tab:tournaments}
\end{table}

In \cite{Milgrom1982} describes how incomplete information can be used to
enhance cooperation, in a similar approach to the proof of the Folk theorem for
repeated games \cite{Maschler2013}. This aspect of incomplete information is
also considered in \cite{Molander1985, Bendor1991, Lee2015} where ``noisy''
tournaments randomly flip the choice made by a given strategy. In
\cite{Murnighan2015} incomplete information is considered in the sense of a
probabilistic termination of each round of the tournament.

As mentioned before, IPD tournaments have been studied in an evolutionary
context: \cite{Ellison1994, Lee2015, Press2012, Stewart2012} consider this in a
traditional evolutionary game theory context. These works investigates
particular evolutionary contexts within which cooperation can emerge as stable.
Often these works consider direct opposition to another strategy and disregard
the population dynamics, for example in \cite{Lee2015} a simple machine learning
algorithm outperforms a strategy found in \cite{Press2012}.

Further to these evolutionary ideas, \cite{Chellapilla1999, DavidB1993} are
examples of using machine learning techniques to evolve particular strategies.
In \cite{Axelrod} Axelrod describes how similar techniques are used to
genetically evolve a high performing strategy from a given set of strategies.
Note that in his original work Axelrod only used a base strategy set of 12
strategies for this evolutionary study. This is noteworthy as
\cite{Axelrod-Pythonprojectteam2015} now boasts over 90 strategies that are
readily available for a similar analysis.

\subsection{Description of the Axelrod python package}\label{sec:description-of-axelrod-python}.

\begin{itemize}
    \item Describe library,
    \item Perhaps include image like:
        http://vknight.org/Talks/2015-10-12-Evolutionary-mathematics/static/outline_of_library.svg
        (am playing with https://pycallgraph.readthedocs.org/en/master/)
    \item Show code to create basic tournament
    \item Show code for a strategy
    \item Show process to contribute strategy
\end{itemize}

\section{Reproducing previous tournaments}\label{sec:reproducing-previous-tournaments}

\section{New strategies, tournaments and implications}\label{sec:new-strategies-and-implications}

\section{Conclusion}\label{sec:conclusion}

\printbibliography
\end{document}
