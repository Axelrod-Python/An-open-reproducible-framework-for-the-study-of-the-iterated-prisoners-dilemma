\documentclass{article}

% Nicer bibliography management
\usepackage[backend=biber, backref=true, firstinits=true, url=true, isbn=true]{biblatex}
\addbibresource{references.bib}
%-----------------------------

% Lazy page formating (we can adjust when submitting)
\usepackage{fullpage}
\usepackage[parfill]{parskip}
%-----------------------------

\title{An open reproducible framework for the study of the iterated prisoner's
dilemma}

\author{Owen Campbell\\
        \and
        Marc Harper\\
        \and
        Vincent Knight\\
        \and
        Karol Langner\\
}

\date{\today}

\begin{document}

\maketitle

\section{Introduction}\label{sec:introduction}

In 1980, Axelrod wrote two papers:~\cite{Axelrod1980a,Axelrod1980b} which
described a computer tournament that has been at the origin of a majority of
game theoretic work~\cite{readallthepapers}.

The tournament is based on an iterated game (see~\cite{Maschler2013} or similar
for details) where two players repeatedly play the normal form game of
(\ref{equ:one-shot}) in full knowledge of each others playing history to date:

\begin{equation}
    \begin{pmatrix}
        3,3 & 0,5\\
        5,0 & 1,1
    \end{pmatrix}
    \label{equ:one-shot}
\end{equation}

The game of (\ref{equ:one-shot}) is called the Prisoner's Dilemma. This is a game with the following rhetoric:

\begin{quote}
    Two players have committed some sort of crime and have been separated for
    questioning, if they \textit{Cooperate} which each other and do not confess
    they will receive a reduced sentence (corresponding to the advantageous utility
    of \(3\) in (\ref{equ:one-shot})). If however, one of them cooperate and the
    other \textit{Defects} then the cooperator will receive a heavier sentence
    (corresponding to a utility of \(0\)) and the defector will be offered a deal
    (corresponding to a utility of \(5\)). If both players defect they both receive
    longer sentences (corresponding to a utility of \(1\)).
\end{quote}

Axelrod's tournaments (and further implementations of these) are sometimes
referred to as Iterated Prisoner's Dilemma tournaments.

\begin{itemize}
\item Review of the tournament itself;
      Original paper by Axelrod and Hamilton~\cite{1981-Axelrod-Hamilton}.
      Some recent discussion of memory one strategies~\cite{press2012iterated, stewart2012extortion}.

\item Discussion about open reproducible science (there are some reference
      around) (Python, git, github etc\dots)
\item Overview of the library (what it can do, what has been done with it)
\item Point at Sections~\ref{sec:reproducing-previous-tournaments}
      and~\ref{sec:new-strategies-and-implications}.
\end{itemize}

\section{Reproducing previous tournaments}\label{sec:reproducing-previous-tournaments}

\section{New strategies, tournaments and implications}\label{sec:new-strategies-and-implications}

\section{Conclusion}\label{sec:conclusion}

\printbibliography
\end{document}
