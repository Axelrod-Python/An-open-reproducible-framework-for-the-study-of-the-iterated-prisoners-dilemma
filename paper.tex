\documentclass{article}

% Nicer bibliography management
\usepackage[backend=biber, backref=true, firstinits=true, url=true, isbn=true]{biblatex}
\addbibresource{references.bib}
%-----------------------------

% Lazy page formating (we can adjust when submitting)
\usepackage{fullpage}
\usepackage[parfill]{parskip}
%-----------------------------

\usepackage{amsmath}
\usepackage{hyperref}
\usepackage{booktabs}

\title{An open reproducible framework for the study of the iterated prisoner's
dilemma}

\author{Owen Campbell\\
        \and
        Marc Harper\\
        \and
        Vincent Knight\\
        \and
        Karol Langner\\
}

\date{\today}

\begin{document}

\maketitle

\section{Introduction}\label{sec:introduction}

As stated in~\cite{Bendor2015}: ``\textit{few works in social science have had
the general impact of [Axelrod's study of the evolution of cooperation]}''.  In
1980, Axelrod wrote two papers:~\cite{Axelrod1980a,Axelrod1980b} which described
a computer tournament that has been at the origin of a majority of game
theoretic work~\cite{Banks1990, Bendor2015, Boyd1987, Chellapilla1999,
    DavidB1993, Doebeli2005, Ellison1994, Gotts2003, Isaac2008, Kraines1989,
    Lorberbaum1994, Milgrom1982, Molander1985, Murnighan2015, Press2012,
Stephens2002, Stewart2012}. As described in~\cite{Bendor2015} this work has not
only had mathematical impact but has also led to insights in biology (for
example in~\cite{Chellapilla1999}, a real tournament where Blu Jays are the
participants is described) and in particular to the study of evolution.

The tournament is based on an iterated game (see~\cite{Maschler2013} or similar
for details) where two players repeatedly play the normal form game of
(\ref{equ:one-shot}) in full knowledge of each others playing history to date.
An excellent description of the \textit{one shot} game is given
in~\cite{Gotts2003} which is paraphrased below:

Two players must choose between \textit{Cooperate} (\(C\)) and \textit{Defect}
(\(D\)):

\begin{itemize}
    \item If both choose \(C\), they receive a payoff of \(R\)
        (\textbf{R}eward);
    \item If both choose \(D\), they receive a payoff of \(P\)
        (\textbf{P}punishment);
    \item If one chooses \(C\) and the other \(D\), the defector receives a
        payoff of \(T\) (\textbf{T}emptation) and the cooperator a payoff of
        \(S\) (\textbf{S}ucker).
\end{itemize}

\begin{equation}
    \begin{pmatrix}
        R,R & S,T\\
        S,S & P,P
    \end{pmatrix}\quad\text{such that } T>R>P>S \text{ and } 2R > T + S
    \label{equ:one-shot}
\end{equation}

The game of (\ref{equ:one-shot}) is called the Prisoner's Dilemma. Numerical
values of \((R,S,T,P)=(3,0,5,1)\) are often used in the literature. Axelrod's
tournaments (and further implementations of these) are sometimes referred to as
Iterated Prisoner's Dilemma tournaments.

\begin{itemize}
\item Review of the tournament itself;
      Original paper by Axelrod and Hamilton~\cite{1981-Axelrod-Hamilton}.
      Some recent discussion of memory one strategies~\cite{press2012iterated, stewart2012extortion}.

\item Discussion about open reproducible science (there are some reference
      around) (Python, git, github etc\dots)
\item Overview of the library (what it can do, what has been done with it)
\item Point at Sections~\ref{sec:reproducing-previous-tournaments}
      and~\ref{sec:new-strategies-and-implications}.
\end{itemize}

\section{Reproducing previous tournaments}\label{sec:reproducing-previous-tournaments}

\section{New strategies, tournaments and implications}\label{sec:new-strategies-and-implications}

\section{Conclusion}\label{sec:conclusion}

\printbibliography
\end{document}
